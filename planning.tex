\documentclass[sigconf,natbib=false]{acmart}
\usepackage[utf8]{inputenc}

\AtBeginDocument{%
  \providecommand\BibTeX{{%
    Bib\TeX}}}

\setcopyright{none}
\copyrightyear{}
\acmYear{2023}
\acmDOI{0000000000}
\acmConference[]{}{}
\acmBooktitle{} 
\acmPrice{}
\acmISBN{}
\usepackage{graphicx}
\graphicspath{ {img} }
\usepackage{caption}
\usepackage{subcaption}
\usepackage{tabularx}
\usepackage{multicol}
\usepackage[style=numeric,sorting=none]{biblatex}
\usepackage{footnote}
\addbibresource{references.bib}
\usepackage{tabularx}
\usepackage{xifthen}
\usepackage{pgfgantt}
\usepackage{adjustbox}
\usepackage[acronym]{glossaries}
\usepackage{datetime}
\usepackage{siunitx}
\renewcommand{\dateseparator}{-}
\ganttset{calendar week text={\currentweek}}
\usepackage{environ}
\usepackage{hyperref}

% ------------------------------------------------------------------------------------------
% Use this if-flag to show or hide the planning content
% ------------------------------------------------------------------------------------------
\newif\ifshow
\showtrue % \showtrue or \showfalse
% ------------------------------------------------------------------------------------------
\ifshow
    \makeglossaries
\fi

\NewEnviron{planning}[1][red]{
  \ifshow
    \textcolor{#1}{\BODY}
  \fi
}

% ------------------------------------------------------------------------------------------
% Use this if-flag to show or hide the placeholder content
% ------------------------------------------------------------------------------------------
\newif\ifplaceholder
\placeholdertrue % \placeholdertrue or \placeholderfalse
% ------------------------------------------------------------------------------------------
\NewEnviron{placeholder}[1][orange]{
  \ifplaceholder
    \textcolor{#1}{\BODY}
  \fi
}

% ------------------------------------------------------------------------------------------
% Use this if-flag to show or hide the planning content
% ------------------------------------------------------------------------------------------
\newif\ifinformation
\informationfalse % \informationtrue or \informationfalse
% ------------------------------------------------------------------------------------------
\NewEnviron{information}[1][violet]{
  \ifinformation
    \textcolor{#1}{\BODY}
  \fi
}

% ------------------------------------------------------------------------------------------

\newcommand\rqf[1]{\paragraph{\textbf{\acrshort{rq#1}}}}
\newcommand\ds[1]{\paragraph{\textbf{\acrshort{ds#1}}}}

\newacronym{pi}{PI}{Phase I}
\newacronym{pii}{PII}{Phase II}
\newacronym{piii}{PIII}{Phase III}


\begin{document}

\title{\begin{placeholder}\end{placeholder}}

\author{Ilnaz Tayebi}
\email{tayebi01@ads.uni-passau.de}
\affiliation{\institution{University of Passau}\city{Passau}\country{Germany}}

\fancyfoot{}
\thispagestyle{empty}
\settopmatter{printacmref=false}
\setcopyright{none}
\renewcommand\footnotetextcopyrightpermission[1]{}
\pagestyle{plain}


\begin{planning}

\section*{Tasks}

\subsection*{\acrlong{pi}}

\begin{itemize}
    \item Install MySql
    \item load data TPC-H
    \item dump data

\end{itemize}

\subsection*{\acrlong{pii}}
\begin{itemize}
    \item prepare presentation
    \item  check if Dump data also sorted
    \item Is B+ Tree or BTree    
\end{itemize}

\subsection*{\acrlong{piii}}
\begin{itemize}
    \item start writing up the Related Work
    \item  get PostgresRAW to work, either compiled directly or via the docker recipe provided, and get it to the point where we can run a smoke test
    \item Write a summary about running the PostgresRaw
    \item investigate: what is the difference between dump in PostgreSQL and in MySQL?
    \item write Abstract
    \item watch Video -> database cracking \cite{google_techtalks_database_2007} and \cite{microsoft_research_database_2016}
    \item finds all related work and state of the arts of database cracking
    \item Write to Prof.Kosch
\end{itemize}

\end{planning}

% \ifshow
%     \pagebreak
% \fi



% \begin{planning}
    

% \end{planning}


\begin{planning}[black]

\begin{figure*}
\centering
\begin{ganttchart}[ 
    vgrid={*{6}{draw=none}, dotted}, 
    hgrid,
    inline,
    x unit=.06cm, 
    y unit title=.8cm, 
    y unit chart=1.2cm,
    title/.append style={shape=rectangle, fill=black!10},
    title height=1,
    time slot format=isodate, 
    time slot format/start date=2024-05-01, 
    today={\the\year-\the\month-\the\day}, 
    today offset=.5,
    progress=today,
    progress label text={},
    bar/.append style={fill=green},
    bar incomplete/.append style={fill=black!10},
    bar progress label anchor=south,
    bar progress label node/.append style={below=0pt},
    group/.append style={draw=black,fill=black},
    group incomplete/.append style={draw=black,fill=black!60},
    group progress label anchor=south,
    group progress label node/.append style={below=0pt},
    milestone inline label anchor=north,
    milestone inline label node/.append style={above=0pt},
    Mile1/.style={milestone/.append style={fill=blue}},
    Mile2/.style={milestone/.append style={fill=red}},
    Mile2/.append style={milestone inline label anchor=east},
    Mile2/.append style={milestone inline label node/.append style={left=5pt}}
    ]{2024-05-12}{2024-11-20} 
\newcommand\textganttbar[5][]{
    \ganttbar[#1, inline,bar label font=\footnotesize]{#2}{#3}{#4}
}
\gantttitlecalendar{year, month=name, week} \\ 
\ganttgroup[inline]{PI:}{2024-05-12}{2024-05-13} \\
\textganttbar[bar/.append style={fill=magenta!50}]{Report I}{2024-05-12}{2024-05-13} \\
\ganttmilestone[Mile1]{Meeting PI}{2024-05-12} \\
\ganttgroup[inline]{PII}{2024-05-12}{2024-05-13} \\
\textganttbar[bar/.append style={fill=purple!50}]{Report II}{2024-05-12}{2024-05-13} \\
\ganttmilestone[Mile2]{Meeting PII}{2024-05-14} \\
\ganttgroup[inline]{PIII}{2024-05-14}{2024-06-17} \\
\textganttbar[bar/.append style={fill=violet!50}]{Report III}{2024-05-14}{2024-07-04} \\
\ganttmilestone[Mile3]{Meeting PIII}{2024-07-04} \\ 
\end{ganttchart} 

\end{figure*}

\end{planning}

    \printglossary[type=\acronymtype]
\end{document}
