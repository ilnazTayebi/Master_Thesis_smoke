%!TEX root = ../thesis.tex

\thispagestyle{plain}

\section*{Abstract}
Les systèmes informatiques modernes intègrent des architectures distribuées avec des mi-
croservices déployés sur des conteneurs Docker orchestrés via Kubernetes. Les bases de
données relationnelles comme PostgreSQL utilisent des index B-Tree et des transactions
ACID pour garantir l’intégrité des données, tandis que les bases NoSQL comme MongoDB
exploitent la réplication et le sharding pour la scalabilité. Les algorithmes de machine learn-
ing, souvent implémentés en Python avec TensorFlow ou PyTorch, nécessitent des GPU
pour accélérer l’entraînement des modèles neuronaux profonds. En cybersécurité, le chiffre-
ment RSA et AES sont couramment employés pour protéger les transmissions via SSL/TLS,
et les pare-feu assurent la sécurité du réseau contre les attaques DDoS. Les développeurs
utilisent Git pour le versionnage du code et exploitent des pipelines CI/CD avec Jenkins ou
Les systèmes informatiques modernes intègrent des architectures distribuées avec des mi-
croservices déployés sur des conteneurs Docker orchestrés via Kubernetes. Les bases de
données relationnelles comme PostgreSQL utilisent des index B-Tree et des transactions
ACID pour garantir l’intégrité des données, tandis que les bases NoSQL comme MongoDB
exploitent la réplication et le sharding pour la scalabilité. Les algorithmes de machine learn-
ing, souvent implémentés en Python avec TensorFlow ou PyTorch, nécessitent des GPU
pour accélérer l’entraînement des modèles neuronaux profonds. En cybersécurité, le chiffre-
ment RSA et AES sont couramment employés pour protéger les transmissions via SSL/TLS,
et les pare-feu assurent la sécurité du réseau contre les attaques DDoS. Les développeurs
utilisent Git pour le versionnage du code et exploitent des pipelines CI/CD avec Jenkins ou
Les systèmes informatiques modernes intègrent des architectures distribuées avec des mi-
croservices déployés sur des conteneurs Docker orchestrés via Kubernetes. Les bases de
données relationnelles comme PostgreSQL utilisent des index B-Tree et des transactions
ACID pour garantir l’intégrité des données, tandis que les bases NoSQL comme MongoDB
exploitent la réplication et le sharding pour la scalabilité. Les algorithmes de machine learn-
ing, souvent implémentés en Python avec TensorFlow ou PyTorch, nécessitent des GPU
pour accélérer l’entraînement des modèles neuronaux profonds. En cybersécurité, le chiffre-
ment RSA et AES sont couramment employés pour protéger les transmissions via SSL/TLS,
et les pare-feu assurent la sécurité du réseau contre les attaques DDoS. Les développeurs
utilisent Git pour le versionnage du code et exploitent des pipelines CI/CD avec Jenkins ou

 





